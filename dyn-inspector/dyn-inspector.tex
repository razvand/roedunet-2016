\documentclass[conference]{IEEEtran}
\usepackage{cite}
\usepackage[pdftex]{graphicx}
\graphicspath{{../figures/}}
\DeclareGraphicsExtensions{.pdf,.jpeg,.png}
\usepackage{amsmath}
\usepackage{algorithmic}
\usepackage{array}
\usepackage{url}
\usepackage{ucs}
\usepackage[utf8x]{inputenc}
\usepackage[english]{babel}
\usepackage{underscore}	  % underscores need not be escaped
\usepackage{subfigure}
\usepackage{verbatim}
\usepackage{float}
\usepackage{booktabs}     % professional tables
\usepackage{parskip}
\usepackage{caption}
\usepackage{hyperref}
\usepackage{amsmath}
\usepackage{mathabx}

\usepackage{color}
\definecolor{mygray}{rgb}{0.5,0.5,0.5}

\usepackage{listings}
\lstset{ %
  basicstyle=\footnotesize\ttfamily,        % the size of the fonts that are used for the code
  breakatwhitespace=false,         % sets if automatic breaks should only happen at whitespace
  breaklines=true,                 % sets automatic line breaking
  captionpos=b,                    % sets the caption-position to bottom
  escapeinside={(*}{*)},           % if you want to add LaTeX within your code
  extendedchars=true,              % lets you use non-ASCII characters; for 8-bits encodings only, does not work with UTF-8
  frame=single,                    % adds a frame around the code
  keepspaces=true,                 % keeps spaces in text, useful for keeping indentation of code (possibly needs columns=flexible)
  numbers=left,                    % where to put the line-numbers; possible values are (none, left, right)
  numbersep=5pt,                   % how far the line-numbers are from the code
  numberstyle=\tiny\color{mygray}, % the style that is used for the line-numbers
  rulecolor=\color{black},         % if not set, the frame-color may be changed on line-breaks within not-black text (e.g. comments (green here))
  showspaces=false,                % show spaces everywhere adding particular underscores; it overrides 'showstringspaces'
  showstringspaces=false,          % underline spaces within strings only
  showtabs=false,                  % show tabs within strings adding particular underscores
  tabsize=2,                       % sets default tabsize to 2 spaces
  title=\lstname                   % show the filename of files included with \lstinputlisting; also try caption instead of title
}

\let\path=\url
\newcommand{\code}[1]{\texttt{\small #1}}
\newcommand{\file}[1]{\texttt{\small #1}}

% correct bad hyphenation here
\hyphenation{op-tical net-works semi-conduc-tor}


\begin{document}

\title{Interactive Inspection of Dynamic Linking and Loading}

\author{\IEEEauthorblockN{Lucia Cojocaru}
\IEEEauthorblockA{Faculty of Automatic Control and Computers\\
University POLITEHNICA of Bucharest\\
Bucharest, 060042\\
Email: lucia.cojocaru@stud.acs.pub.ro}
\and
\IEEEauthorblockN{Răzvan Deaconescu}
\IEEEauthorblockA{Faculty of Automatic Control and Computers\\
University POLITEHNICA of Bucharest\\
Bucharest, 060042\\
Email: razvan.deaconescu@cs.pub.ro}}

% make the title area
\maketitle

\begin{abstract}
Students learn about the existence of the linking process, but few know and understand what it really does. The theoretical explanation often requires a good background in compilers and operating systems, but it can be easier to explain with a proper example that shows every change of the process. While the other available tools are not beginner friendly, ELF Detective is, and by using any binary files, even a personal ones that are well known for the user, it can thoroughly analyse and explain in an educational way the linking phase. This makes it a lot easier to understand without having a very good technical background. This is the tool to use for a more educational perspective of the linking stage.
\end{abstract}

\section{Introduction}
\label{sec:introduction}
The linking stage is one of the most important topics in computer science and represents a cornerstone for every individual training in this area. While at first is easy to feel like this process is easy to understand, one might find himself to a point where the lack of a good knowledge of this stage led to complications.

The problem is not as obvious as one not knowing what happens during the process, but the lack of a more practical experience, which is hard to gain. Ideally, we could learn about this topic in a less theoretical way, that can form a better background for an individual of any level of expertise. 

Doing a manual analysis of the transformations that happen during the linking stage will only lead to  a whole of machine code that gets merged. This is not very helpful since the changes are almost unrecognizable, leading to confusion for an unexperienced user.

The project that we present in this paper, ELF Detective, is an educational tool created with the goal of helping students understand the changes that occur in the linking process. It offers an interactive interface where it presents the important data as a comparison between its states in the executable and object files. To find any additional information, the user has to click on any item and it will show the information in an easy to read format that clarifies the "how's" and the "what's" regarding the changes.

\subsection{Motivation}
\label{sec:motivation}

We base our  motivation for this project on the fact that many complex topics, in this case the linking process, are a lot harder to understand without a practical example, and even in the case of one being provided, most often it's too generic. To our knowledge, there are no educational tools allowing an easy exploration of the linking process, and any other program that might help is not beginner friendly.

\subsection{Objective}
\label{sec:obj}

Our main objective with this project is to help others understand the linking process better while spending a lot less time. Since there is no other tool that we know of that has this focus, the one we were to create had to be easy understand.

From an education point of view, this project must be able to explain the changes that occur in the linking stage in the easiest way possible so a student of any background can understand it well. This means that all addresses must be explained, any link between the executable and object files is well represented and the wording of the explanation is not too complex.

Our second objective for this project is to be interactive. This means that not only the program will have a good response time, but the GUI will seem straightforward to use. If the user is currently inspecting a project and require more information, he just has to click on it. This works for both symbols and code lines, and the output not only that explains the requested information as best as it can, but it does it for both the executable file and the object file it came from.

Lastly the project must be correct, have a good response time (under a second) and scale well. These topic is covered later on.


\section{Related Work}
\label{sec:related-work}
At present, in order to make a preliminary assessment of how the linker and loader work, there are tools which make it possible to examine the contents of an executable before it is run. 

\subsection{Basic Unix Tools}

On Unix systems, basic command line tools such as \textit{readelf}\footnote{\url{https://sourceware.org/binutils/docs/binutils/readelf.html}} and \textit{objdump}\footnote{\url{https://sourceware.org/binutils/docs/binutils/objdump.html}} achieve the functionality described above for ELF executables. They can display the sections of a program with their corresponding offsets within the program address space. This can be useful as the loader can keep this arrangement, although this is not guaranteed.

However, they cannot offer detailed information, or none at all, about particular sections of interest in the dynamic linking and loading process, such as the \textit{.plt} or \textit{.got.plt} sections.

In order to properly understand how the linker and loader function and interact, tools which have the possibility to instrument an executable are necessary. At present, these tools are debuggers.

\subsection{GDB}

Unix platforms provide by default \textit{gdb} which is a powerful debugging tool for ELF executables. \textit{Gdb} includes standard commands for displaying the shared libraries used by the program, as well as the offsetted locations of the segments in the running ELF.

However, in order to observe the detailed process of dynamic linking and loading, the user must manually interact with \textit{gdb}. For example, in the case of a function call to a shared library,  the workflow involves setting breakpoints and obtaining the address of the \textit{.plt} section and \textit{.got.plt} entry for function.

This implies that knowledge of the process is a prerequisite in order to follow the subtle changes happening at the time of dynamic linking.

\subsection{LLDB}

Despite its powerful features, \textit{gdb} lacks integration with other software and cannot be included as a module or library in custom programs. It is designed as a manually operated command line tool for the main purpose of debugging.

For this reason, the LLVM project\footnote{\url{http://llvm.org/}} has started developing a high performance debugger which broadly covers the existing \textit{gdb} functionality. \textit{Lldb}\footnote{\url{http://lldb.llvm.org/}} supports command-line debugging, but can also be included as a module in scripting languages such as Python\footnote{\url{https://www.python.org/}}.

Therefore, in order to observe the linking and loading process, \textit{lldb} offers a similar set of steps as \textit{gdb} through its command line feature. However, its main advantage is that its functionality can easily be extended to fit more specific needs.


\section{Requirements and Design Principles}
\label{sec:design}
The tools described in the previous section allow visualisation of the linking and loading processes as long as the user is aware of what needs to be followed. There are two main drawbacks of the usage of debuggers as a means to understand linking and loading. The first is that debuggers do not highlight any process in particular. They are tools with no purpose until the user gives them one. The second is that debuggers are very complex and offer a multitude of functions. This discourages inexperienced users from using them.

Firstly, stepping in a debugger is not particularly useful for somebody who first encounters these concepts and needs a hand-on application to fully grasp what is happening at each stage. Debuggers cannot be used as a learning tool properly, as they do not offer pointers as to what is happening in the linking and loading processes. Previous knowledge is required.

Secondly, in order to use specific debuggers, the user must be accustomed to their functionality. Both \textit{gdb} and \textit{lldb} have a wide range of commands which make them very powerful tools, at the cost of user experience.

Prior to using them, one must learn what the proper commands are for what is intended, as well as what to understand from the output returned. For an inexperienced user whose goal is to understand the mechanics of dynamic linking and loading, this can be a time consuming and unnecessary process.

\subsection{Objectives}

The preceding section underlines the issues with the existing approaches to understanding how the dynamic loader and linker work. Despite providing extensive functionality, debuggers are not a solution to the specific problem of inspecting the dynamic linker and loader, but rather a possible workaround. 

This project fills the need for a dedicated solution for analysing the dynamic linking and loading processes. DynInspector is a tool which provides full functionality for exploring dynamic linking and loading after an ELF executable is launched. It consists of two major parts:
\begin{itemize}
\item an extension to the \textit{lldb} python module which implements dedicated methods for analysing the behaviour of the dynamic linker and loader;
\item a user-friendly graphical interface which isolates the necessary functionality for interacting with the program during dynamic linking and loading;
\end{itemize}

\subsection{Design}

DynInspector focuses the capabilities of a debugger, in this case \textit{lldb}, on the observation of and interaction with the dynamic linking and loading process. The focus is achieved through a set of design principles:

\begin{itemize}
\item \textbf{Create a user-friendly graphical interface.} The user should only interact with the tool through an intuitive and minimal graphical user interface. Unnecessary functionality should not be provided.
\item \textbf{Ensure an interactive user experience.} The user should be able to control the target executable being inspected by means of the gui and nothing else. The GUI should provide the necessary functionality and be responsive at all times.
\item \textbf{Extend the functionality of existing tools.} The functionality desired in order to observe the dynamic loader should be contained in a stand-alone module. The code should be reusable as a python module and should extend, not replicate, the existing \textit{lldb} functionality.
\end{itemize}


\section{Implementation Details}
\label{sec:implementation}
To achieve the desired functionality, the DynInspector tool is designed and built upon the following software support and hardware specifications:

\begin{itemize}
\item \textbf{OS platform}: Ubuntu 14.04  LTS
\item \textbf{OS type}: 32 bit / i686
\item \textbf{Target executable format}: ELF 32-bit LSB  executable, Intel 80386
\item \textbf{Programming language}: \textbf{python 2.7}
\item \textbf{Graphical User Interface}: \textbf{PySide} python module \textbf{1.2.1}
\begin{itemize}
\item Based on \textbf{QtCore 4.8.6}
\end{itemize}
\item \textbf{lldb}: \textbf{lldb} python module \textbf{3.6}
\item \textbf{Dependencies}: \textbf{llvm-3.6} and \textbf{clang 3.6}
\end{itemize}

The DynInspector tool is written in the Python scripting language. It is structured on two Python threads, one for the gui and one for background computations. The choice for using two threads is justified by the following two reasons:
\begin{itemize}
\item The gui is required to run on a dedicated thread in order to be responsive. The other computations require a background thread.
\item There are no intensive background computations which makes one thread alone suitable for the task.
\end{itemize}

\subsection{Graphical User Interface}

The gui is implemented using Python PySide bindings to Qt. It is important that the gui differentiates between dynamic linking, along with lazy binding, thus there are two different application modes which I will detail below.

\paragraph{Dynamic Linking / Lazy Binding Mode.}

This mode highlights the dynamic linking process, more specifically address resolution with the lazy binding mechanism. Basically it shows what happens when a function from a dynamically shared library is called from a program.

\paragraph{Dynamic Loading Mode.}

This mode highlights memory mappings for shared libraries and presents the user initiated function call mechanism for functions in shared libraries. Basically it traces \textit{dlopen}, \textit{dlsym} and \textit{dlclose} calls.

Overall, the gui has a simple and intuitive aspect which allows the user to use it without previous experience. Moreover, it offers all the functionality necessary for analysing the dynamic linking and dynamic loading processes with a minimum complexity by use of basic interactive widgets.

\subsection{Background Work}

In this particular case, background computations consist of:
\begin{itemize}
\item \textbf{Handling a debugger instance which runs an ELF executable.} This means sending commands to it and obtaining status information about the process being run.
\item \textbf{Processing the information from the debugger instance.} The purpose of the current application is to provide selective information to the user. The parsed data is forwarded to the gui thread.
\end{itemize}

The background thread acts as an intermediate between the gui and \textit{lldb} debugger.

\subsection{Communication and Synchronization}

The applications threads communicate using Python \textit{Signals and Slots} from \textit{PySide}. It is a mechanism for user-defined callbacks. Each thread registers, or \textit{connects}, a \textit{slot} function to a \textit{signal} object. The purpose of this function is to handle a certain event. For the signal, the event is \textit{emitted} on demand.

The response of the background thread are instantaneous due to the performance of the \textit{lldb} module and, thus, further synchronization is not necessary at this stage.

\subsection{LLDB Wrapper Module}

The background thread creates and interacts with an instance of the \textit{lldb} debugger. The instance is not used directly, but through a wrapper module which extends the functionality of the Python \textit{lldb} module.

This module offers support includes support for:
\begin{itemize}
\item \textbf{Printing the current frame}
\item \textbf{Reading Global Offset Table entries and PLT entries}
\item \textbf{Providing information about shared libraries} mapped in the program memory (modules)
\item \textbf{Creating breakpoints for the functions with stubs in the \textit{.plt} and for \textit{dlopen}, \textit{dlsym}, \textit{dlclose} calls}
\item \textbf{Execution handling}: run, continue, step. This limits the points where the program is stopped to events of interest for the linker and loader.
\end{itemize}

The main advantage of isolating the above functionality in a stand-alone module is that it is independent of the current application and can be imported in other projects as well. The background thread aggregates the wrapper module and interacts with it by calling its methods.


\section{Feedback and Evaluation}
\label{sec:evaluation}
The DynInspector tool is intended for both first and second year computer science students and hobbyists who are interested in understanding the functionality of dynamic linkers and loaders. The purpose of the tool is primarily educational.

For this reason, the testing plan involved approaching the target users and allowing them to run and test the application.

\subsection{User Interface}

In order to asses if the interface satisfies the objectives, we evaluated the users' difficulty in using the interface and understanding what each component, or widget, does.

After the first version of DynInspector was tested, initial feedback remarked that the gui design was primitive and discouraged users from using the tool. We updated the GUI to have a similar design to that of IDEs such as \textit{Eclipse}\footnote{\url{https://eclipse.org/}} in order to give the user a familiar sensation when using the tool. The tool become more visually appealing with these changes.

Another feedback point was that users were not aware that the application has two inspection modes. In order to make it simple and clear for all users, regardless of their operating system, we moved the button to the \textit{Status and Control Bar}. The ten users who tested the application in its final form were able to switch the application mode without being told about the mode beforehand and without help in locating the widget.

Users remarked that it is difficult to follow the program execution flow through the assembly code alone. This is because most of them take assembly courses in their second year. For this reason, we added an additional \textit{Source Code} tab window to the \textit{Assembly Code} tab window, functional only for executables compiled with debug symbols. It shows the content of the source file and also highlights the current instruction. The users can now look at both the assembly and source code which makes it easier to understand where the program is and what is happening.

The changes described above resulted in a positive final feedback. Users described the gui of DynInspector as being intuitive and straight-forward. This indicates that the objective of creating an easy-to-use tool was achieved.

\subsection{Tool Usage}

In the previous chapters we argued that DynInspector was designed to be intuitive and self-explanatory and that an inexperienced user should be able to use it without previous knowledge and without assistance.

Overall feedback revealed that despite the straight-forward use of the interface, five out of twelve users were not certain what they were expected to do at first glance. They mentioned that only after a couple of consecutive runs did they understand what the tool was doing. To solve this problem, we created the \textit{Help} button which offers a crash-course on the theory of dynamic linking and loading, as well as a short description of what each application mode does. As a result, users instinctively clicked \textit{Help} and read the instructions when they first used the program. They figured out more easily what they were expected to do with the tool, where to look and how to interpret the data displayed.

The simple and intuitive design of the gui helped users figure out how to use the tool in the majority of situations. The feedback received on tool usage was tackled with a short tutorial on dynamic linking and loading and this proved to be an effective solution.

\subsection{Educational Purpose}

The most important objective of the project is to create a simple and straight-forward inspection tool for dynamic linking and loading which is adequate for inexperienced users interested in learning about these processes. In order to asses if DynInspector properly emphasises the essential aspects of dynamic linking and loading, we have analysed the feedback received from users.

Initial feedback revealed that even though users could easily use the tool, they did not understand what dynamic linking and dynamic loading were. For example, one of the first testers mentioned he had opened twenty browser tabs with information on dynamic linking and loading in order to fully understand what was happening. This problem was fixed when we created the \textit{Help} section which offered a crash-course on dynamic linking, dynamic loading and lazy procedure linkage.

The most recent feedback revealed that all the users who reviewed the DynInspector tool agreed that what is happening is simple and clear. They properly distinguished between dynamic linking and dynamic loading by using the two different inspection modes. The \textit{Help} manual offers a basic theoretical understanding of the process, whereas the inspector itself helps deepen the knowledge with a hand-on tutorial on dynamic linking and dynamic loading.

Moreover, ten out of twelve users confirmed that they would not have been able step through the entire process by themselves, without using the tool and with no previous knowledge, as it was too complicated. There are two main reasons they presented. The first one is that they are not familiar enough with debugger tools such a \textit{gdb} and, in most cases, the complexity of the command line debugger discouraged them from attempting to use it. It is important to notice that this feedback belongs not only to first year students, but also to third year ones. Secondly, they had no knowledge of the dynamic linking and dynamic loading processes and, as a result, they did not know what they should look at in the debugger output and what each step meant. This validates the motivation for creating this project and the final positive feedback supports that the main objectives have been achieved.
 
\subsection{Additional Feedback}

Aside from the main feedback points which validate the main purpose of the project, users have presented valuable feedback regarding the functionality of the project and compatibility with their systems.

Feedback revealed that seven out of twelve users had 64 bit systems which made the DynInspector tool unusable for them. In order to test the tool, they had to use a virtualization solution such a VmWare\footnote{\url{http://www.vmware.com/}} and install a 32 bit machine. This process is cumbersome and might discourage users who do not have virtualization software already installed on their system from using the tool. For this reason, most users requested that further development includes support for 64 bit machines.
Overall, the feedback revealed that DynInspector has made an impact in the sense that its goal of presenting dynamic loading and dynamic linking in an accessible manner to inexperienced users has been completed.


\section{Conclusion and Further Work}
\label{sec:conclusion}
In this article I presented DynInspector which is an educational tool for analysing the obscure process of dynamic linking and loading. DynInspector has a user-friendly user interface and allows an interactive inspection of ELF executables. The project has created a positive user experience and has achieved its purpose of transmitting relevant knowledge. However feedback revealed that there are further issues to be addressed in order to make the tool accessible for users on various other platforms.

The tool was evaluated by second and third year students primarily and the feedback received helped assess the quality of the software and improve user experience. The users were able to use the gui without additional help. The intuitive design and placement of the widgets produced an effortless interaction between the user and the application. While the console messages provided descriptive information about each step, the various tables, code view windows and buttons gave the user a hands-on learning experience.

Most importantly, users assessed the tool positively regarding its educational purpose. They understood what dynamic linking and what dynamic loading were hands-on and gained detailed knowledge, rather than theoretical concepts alone. They stated that the mechanisms were straightforward and the process overall was easy to understand using DynInspector . This underlines the fact that the key purpose of an educational tool was achieved and that a rather obscure mechanism was presented in a clear and easy manner.

Further development of DynInspector should focus the main issues addressed by user feedback. After that, work on self-standing features should be considered. Below I will detail the main areas of future development.

\subsection{Support for 64 bit Linux Systems}

DynInspector should offer support for 64 bit Linux systems in order to be used by a wider range of users. The current implementation is valid for 32 bit machines only. One of the main setbacks is that \textit{lldb-3.6} does not properly extract the symbol stubs for the functions in the \textit{.plt} section for 64 bit systems. An approach to solving this issue is switching to a more recent \textit{lldb} version.

\subsection{Support for Parallel Execution}

At present, DynInspector only supports inspection for single threaded executables. The tool extracts an instance of the first thread from the process of the executable. This thread is then used to obtain various information such as frame assembly code, program counter and so on. 

Support for multi-threaded applications must tackle issues such as the correct management of all threads within the \textit{lldb} wrapper module, as well as thread synchronization during the debugging session.

\subsection{ARM Support}

Currently DynInspector can analyse the dynamic linking and loading process for the x86 architecture. It would be useful to extend support to various other architectures such as ARM.

There are two main reasons why the current DynInspector implementation does not offer ARM support. Firstly, ARM defines its own ISA \cite{arm-manual} which is fundamentally different from the x86 one. ARM is a \textit{RISC} architecture, whereas x86 is basically \textit{CISC}. Secondly, ARM compilers generate a different executable format which cannot be analysed in the same way as its x86 counterpart. Being a low-level inspection tool, DynInspector relies on a very strict instruction order for the assembly code. Any minor difference may result in unexpected behaviour.


\bibliographystyle{abbrv}
\bibliography{dyn-inspector}

\end{document}
