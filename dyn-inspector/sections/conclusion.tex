In this article we present DynInspector, an educational tool for analysing the process of dynamic linking and loading. DynInspector has a user-friendly user interface and allows an interactive inspection of ELF executables. The project has created a positive user experience and has achieved its purpose of transmitting relevant knowledge. However feedback revealed that there are further issues to be addressed in order to make the tool accessible for users on various other platforms.

The tool was evaluated by second and third year students primarily and the feedback received helped assess the quality of the software and improve user experience. The users were able to use the gui without additional help. The intuitive design and placement of the widgets produced an effortless interaction between the user and the application. While the console messages provided descriptive information about each step, the various tables, code view windows and buttons gave the user a hands-on learning experience.

Most importantly, users assessed the tool positively regarding its educational purpose. They understood what dynamic linking and what dynamic loading were hands-on and gained detailed knowledge, rather than theoretical concepts alone. They stated that the mechanisms were straightforward and the process overall was easy to understand using DynInspector . This underlines the fact that the key purpose of an educational tool was achieved and that a rather obscure mechanism was presented in a clear and easy manner.

Further development of DynInspector will focus the issues addressed by the user feedback. After that, work on self-standing features should be considered.

DynInspector should offer support for 64 bit Linux systems in order to be used by a wider range of users. The current implementation is valid for 32 bit machines only. One of the main setbacks is that \textit{lldb-3.6} does not properly extract the symbol stubs for the functions in the \textit{.plt} section for 64 bit systems. An approach to solving this issue is switching to a more recent \textit{lldb} version. Further more, we plan to extend support to various other architectures such as ARM, with its own specific ISA~\cite{arm-manual}, from the current support of the x86 architecture only.
