In this article I presented DynInspector which is an educational tool for analysing the obscure process of dynamic linking and loading. DynInspector has a user-friendly user interface and allows an interactive inspection of ELF executables. The project has created a positive user experience and has achieved its purpose of transmitting relevant knowledge. However feedback revealed that there are further issues to be addressed in order to make the tool accessible for users on various other platforms.

The tool was evaluated by second and third year students primarily and the feedback received helped assess the quality of the software and improve user experience. The users were able to use the gui without additional help. The intuitive design and placement of the widgets produced an effortless interaction between the user and the application. While the console messages provided descriptive information about each step, the various tables, code view windows and buttons gave the user a hands-on learning experience.

Most importantly, users assessed the tool positively regarding its educational purpose. They understood what dynamic linking and what dynamic loading were hands-on and gained detailed knowledge, rather than theoretical concepts alone. They stated that the mechanisms were straightforward and the process overall was easy to understand using DynInspector . This underlines the fact that the key purpose of an educational tool was achieved and that a rather obscure mechanism was presented in a clear and easy manner.

Further development of DynInspector should focus the main issues addressed by user feedback. After that, work on self-standing features should be considered. Below I will detail the main areas of future development.

\subsection{Support for 64 bit Linux Systems}

DynInspector should offer support for 64 bit Linux systems in order to be used by a wider range of users. The current implementation is valid for 32 bit machines only. One of the main setbacks is that \textit{lldb-3.6} does not properly extract the symbol stubs for the functions in the \textit{.plt} section for 64 bit systems. An approach to solving this issue is switching to a more recent \textit{lldb} version.

\subsection{Support for Parallel Execution}

At present, DynInspector only supports inspection for single threaded executables. The tool extracts an instance of the first thread from the process of the executable. This thread is then used to obtain various information such as frame assembly code, program counter and so on. 

Support for multi-threaded applications must tackle issues such as the correct management of all threads within the \textit{lldb} wrapper module, as well as thread synchronization during the debugging session.

\subsection{ARM Support}

Currently DynInspector can analyse the dynamic linking and loading process for the x86 architecture. It would be useful to extend support to various other architectures such as ARM.

There are two main reasons why the current DynInspector implementation does not offer ARM support. Firstly, ARM defines its own ISA \cite{arm-manual} which is fundamentally different from the x86 one. ARM is a \textit{RISC} architecture, whereas x86 is basically \textit{CISC}. Secondly, ARM compilers generate a different executable format which cannot be analysed in the same way as its x86 counterpart. Being a low-level inspection tool, DynInspector relies on a very strict instruction order for the assembly code. Any minor difference may result in unexpected behaviour.
