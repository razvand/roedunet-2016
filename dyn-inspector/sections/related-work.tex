At present, in order to make a preliminary assessment of how the linker and loader work, there are tools which make it possible to examine the contents of an executable before it is run. 

\subsection{Basic Unix Tools}

On Unix systems, basic command line tools such as \textit{readelf}\footnote{\url{https://sourceware.org/binutils/docs/binutils/readelf.html}} and \textit{objdump}\footnote{\url{https://sourceware.org/binutils/docs/binutils/objdump.html}} achieve the functionality described above for ELF executables. They can display the sections of a program with their corresponding offsets within the program address space. This can be useful as the loader can keep this arrangement, although this is not guaranteed.

However, they cannot offer detailed information, or none at all, about particular sections of interest in the dynamic linking and loading process, such as the \textit{.plt} or \textit{.got.plt} sections.

In order to properly understand how the linker and loader function and interact, tools which have the possibility to instrument an executable are necessary. At present, these tools are debuggers.

\subsection{GDB}

Unix platforms provide by default \textit{gdb} which is a powerful debugging tool for ELF executables. \textit{Gdb} includes standard commands for displaying the shared libraries used by the program, as well as the offsetted locations of the segments in the running ELF.

However, in order to observe the detailed process of dynamic linking and loading, the user must manually interact with \textit{gdb}. For example, in the case of a function call to a shared library,  the workflow involves setting breakpoints and obtaining the address of the \textit{.plt} section and \textit{.got.plt} entry for function.

This implies that knowledge of the process is a prerequisite in order to follow the subtle changes happening at the time of dynamic linking.

\subsection{LLDB}

Despite its powerful features, \textit{gdb} lacks integration with other software and cannot be included as a module or library in custom programs. It is designed as a manually operated command line tool for the main purpose of debugging.

For this reason, the LLVM project\footnote{\url{http://llvm.org/}} has started developing a high performance debugger which broadly covers the existing \textit{gdb} functionality. \textit{Lldb}\footnote{\url{http://lldb.llvm.org/}} supports command-line debugging, but can also be included as a module in scripting languages such as Python\footnote{\url{https://www.python.org/}}.

Therefore, in order to observe the linking and loading process, \textit{lldb} offers a similar set of steps as \textit{gdb} through its command line feature. However, its main advantage is that its functionality can easily be extended to fit more specific needs.
