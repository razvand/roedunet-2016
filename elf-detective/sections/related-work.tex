There are many applications that offer a great overview of binary files, but each of them have a different purpose. To our knowledge, ELF Detective is the first of its kind because it can handle ELF files just as well as any other existing program, but its focus is to find connections between files that may have been changed during the linking stage and to provide an educational insight of these changes. Because of this, it is more relevant that we only present tools that offer similar functionalities.

Presenting the tools of the trade shows the need for a project like the one we present in this paper. These tools are old and require human reasoning to completely understand their output, since they present a lot of information, but in a raw, terminal based format.

The most generic tools that handle ELFs are: \textbf{objdump}\footnote{\url{https://sourceware.org/binutils/docs/binutils/objdump.html}}, \textbf{readelf}\footnote{\url{https://sourceware.org/binutils/docs/binutils/readelf.html}}, \textbf{nm}\footnote{\url{https://sourceware.org/binutils/docs/binutils/nm.html}}, \textbf{addr2line}\footnote{\url{https://sourceware.org/binutils/docs/binutils/addr2line.html}}, tools provided by the same package (binutils), which rarely gets an update. These tools are revised only for bug fixes or tweakings of the display format. As a result of their age they're heavily used and well known, but due to the shortage of updates over the time they became outmoded. They have a minimum level of interactivity, which means that based on the what we ask for, they answer with some raw data and nothing more. This is not really an issue for more advanced users, but they are less appreciated by anyone that wants to get better at understanding ELFs and, even if you have a good knowledge of how they work, you still need to analyse the output and connect the dots in between to find the answer you were looking for. These tool do not allow to inspect multiple files at once and will show incomplete information when it comes to external symbols and even some function calls.

Alongside these generic tools, there are plenty more specific programs that work very well, but only implement a specific subset of functionalities. These are tools with a high level of interactivity, but they do not serve any educational purpose since they are mainly used for debugging, cracking and reverse engineering. A few of these tools are:

\begin{itemize}  
	\item Hex editor: \textbf{Bless}\footnote{\url{http://home.gna.org/bless/}}, \textbf{wx Hex Editor}\footnote{\url{http://www.wxhexeditor.org/}}
	\item Disassembly and reverse engineering: \textbf{Radare 2}\footnote{\url{http://radare.org/r/}}, \textbf{IDA}\footnote{\url{https://www.hex-rays.com/products/ida/}}
\end{itemize}